Knihovna (C++) umožňuje přístup k hodnotám voleb v souboru (.ini).

\subsection*{Nástin použítí\+:}

Uživatel vytvoří „\+Configuration“ a předá ji jméno souboru. Uživatel vytvoří „\+Section“. Dalším krokem je specifikace voleb v sekci (instance „\+Option\+Proxy“,či „\+List\+Option\+Proxy“). Jejich hodnoty si pak uživatel uloží do vlastních proměnných/atributů. Při ukládání konfugurace jsou potom hodnoty v "Option\+Proxy“ uloženy také.

\subsection*{Mini příklad\+:}



 
\begin{DoxyPre}{\ttfamily 
Configuration config; 
config.open(inpuStream); // nyni se nactou data z daneho srteamu}\end{DoxyPre}



\begin{DoxyPre}{\ttfamily Section optionalSec = config.SpecifySection("optionalSection"); 
toto najde blok dat odpovidajici sekci nazvane "optionalSection"}\end{DoxyPre}



\begin{DoxyPre}{\ttfamily OptionProxy<StringSpecifier> optionFileName =  
            dummySection.SpecifyOption<StringSpecifier>("Save File", StringSpecifier(), "defaultFileName" );
nyni muze uzivatl prostrednictvim saveFileName cist a menit tuto volbu.}\end{DoxyPre}



\begin{DoxyPre}{\ttfamily ...}\end{DoxyPre}



\begin{DoxyPre}{\ttfamily optionFileName.Set("newFile.sav");}\end{DoxyPre}



\begin{DoxyPre}{\ttfamily ...}\end{DoxyPre}



\begin{DoxyPre}{\ttfamily cofing.save(outputStream); // timto se vsechny volby ulozi
}\end{DoxyPre}
 



\subsection*{Hlavní vlastnoti\+:}

Datové typy (podle kterých se mají textová data interpretovat) jsou určeny pomocí specialních tříd tzv.\char`\"{}\+Type\+Specifiers\char`\"{} (v příkladu \char`\"{}\+String\+Specifier\char`\"{}). Vlastní implementace těchto tříd umožuje uživateli definovát vlastní přechodové funkce mezi textem a požadovanými daty. Signatura konkrétního typu určeného takovou třídou je dostupná v \char`\"{}\+Type\+Speciers\+::\+Value\+Type\char`\"{}.

Třída „\+Configuration“ a „\+Option\+Proxy“ spolu komunikují po celou dobu. Po ukončení práce lze všechny aktuální hodnoty v specifikovaných \char`\"{}\+Option\+Proxy“ uložit pomocí \char`\"{}Configuration.\+save\char`\"{}. Konfliktům je zamezeno
tím, že od každé volby je vydána maximálně jedna instance \char`\"{}Option\+Proxy".

\subsection*{Příklad rendereru\+:}






\begin{DoxyPre}{\ttfamily 
class Upbp
\{
public:
    Upbp(Configuration& configuration)  
    \{
        Section sectionUpbp = configuration.SpecifySection("UPBP", Requirement::MANDATORY);
        beamDensity\_ = sectionUpbp.SpecifyOption("BeamDensity", FloatSpecifier(), 15.f);
    \}
    ...
\};
class Vcm
\{
public:
    Vcm(Configuration& configuration)
    \{
        Section sectionVcm = configuration.SpecifySection("VCM", Requirement::MANDATORY);
        radius\_ = sectionVcm.SpecifyOption("Radius", FloatSpecifier(), 15.f);
    \}
    ... 
\};}\end{DoxyPre}



\begin{DoxyPre}{\ttfamily enum AlgorithmType
\{
    UPBP,
    VCM
\}}\end{DoxyPre}



\begin{DoxyPre}{\ttfamily Configuration configuration("render.ini", InputFilePolicy::IGNORE\_NONEXISTANT);}\end{DoxyPre}



\begin{DoxyPre}{\ttfamily Section mainSection = configuration.SpecifySection("Main", Requirement::MANDATORY);
auto algorithmType = mainSection.SpecifyOption("Algorithm", EnumSpecifier<AlgorithmType>(\{ \{ "VCM", VCM \}, \{ "UPBP", UPBP \} \}), AlgorithmType::VCM);
switch (algorithmType.Get())
\{
    case VCM: renderer = new Vcm(configuration); break;
    case UPBP: renderer = new Upbp(configuration); break;
\}
}\end{DoxyPre}




\subsection*{Příklad jednoduché hry\+:}


\begin{DoxyPre}{\ttfamily 
Hra hadani nahodneho cisla v urcitem rozsahu. 
class Game
\{
    const int MAX\_GUESSING\_NUMBER = 10000;
    const int STANDARD\_RANGE\_START = 0;
    const int STANDARD\_RANGE\_END = 100;
public:
    Game(Configuration& configuration)
    \{
        Section gameSection = configuration.SpecifySection("Game", Requirement::OPTIONAL);
Nacteni rozsahu.
        rangeStart\_ = gameSection.SpecifyOption(
            "RangeStart",
            IntegerSpecifier(1, MAX\_GUESSING\_NUMBER),
            STANDARD\_RANGE\_START,
            Requirement::MANDATORY);
        rangeEnd\_ = gameSection.SpecifyOption(
            "RangeEnd",
            IntegerSpecifier(1, MAX\_GUESSING\_NUMBER), // zde neni pouzit rangeStart\_.Get() jako dolni mez, protoze 1) v zavislosti na volajicim v nem jeste nemusi byt spravna hodnota, 2) pri jeji zmene (v programu) by spodni rozsah neaktualizoval
            STANDARD\_RANGE\_END,
            Requirement::MANDATORY);
Nacteni zprav vypisovanych uzivateli (v podstate lokalizace).
        higher\_ = gameSection.SpecifyOption(
            "Higher",
            StringSpecifier(),
            "Guess too low.");
        lower\_ = gameSection.SpecifyOption(
            "Lower",
            StringSpecifier(),
            "Guess too high.");
        outOfBounds\_ = gameSection.SpecifyOption(
            "OutOfBounds",
            StringSpecifier(),
            "Guess is not within the allowed bounds.");
        victoryMessage\_ = gameSection.SpecifyOption(
            "VictoryMessage",
            StringSpecifier(),
            "Victory!");
        defeatMessage\_ = gameSection.SpecifyOption(
            "DefeatMessage",
            StringSpecifier(),
            "Defeat...");
    \}}\end{DoxyPre}



\begin{DoxyPre}{\ttfamily     bool Guess(int value)
    \{
Kontrola rozsahu
        if (value < rangeStart\_.Get() || value > rangeEnd\_.Get())
        \{
            std::cout << outOfBounds\_.Get() << std::endl;
            return false;
        \}}\end{DoxyPre}



\begin{DoxyPre}{\ttfamily          ... 
    \}}\end{DoxyPre}



\begin{DoxyPre}{\ttfamily     void SetRange(int start, int end)
    \{
Nastaveni rozsahu v konfiguraci.
        rangeStart\_.Set(start);
        rangeEnd\_.Set(end);
    \}}\end{DoxyPre}



\begin{DoxyPre}{\ttfamily     void ReportResult(bool win)
    \{
Vyhral nebo prohral.
        std::cout << (win ? victoryMessage\_.Get() : defeatMessage\_.Get()) << std::endl;
    \}}\end{DoxyPre}



\begin{DoxyPre}{\ttfamily private:
    Option<IntegerSpecifier> rangeStart\_;
    Option<IntegerSpecifier> rangeEnd\_;
    Option<StringSpecifier> higher\_;
    Option<StringSpecifier> lower\_;
    Option<StringSpecifier> outOfBounds\_;
    Option<StringSpecifier> victoryMessage\_;
    Option<StringSpecifier> defeatMessage\_;
\};}\end{DoxyPre}



\begin{DoxyPre}{\ttfamily class Sound
\{
public:
    Sound(Configuration& configuration)
    \{
        Section soundSection = configuration.SpecifySection("Sound"); // implicitne optional 
        IntegerSpecifier volumeSpecifier = IntegerSpecifier(0, 100);
Nabindovani hlasitosti.
        musicVolume\_ = soundSection.SpecifyOption("Music", volumeSpecifier, 100);
        voiceVolume\_ = soundSection.SpecifyOption("Voice", volumeSpecifier, 100);
        effectVolume\_ = soundSection.SpecifyOption("Effects", volumeSpecifier, 100);
    \}}\end{DoxyPre}



\begin{DoxyPre}{\ttfamily Primy pristup k nastavovani hlasitostem volajicim.
    Option<IntegerSpecifier>\& MusicVolume()
    \{
        return musicVolume\_;
    \}
    Option<IntegerSpecifier>\& VoiceVolume()
    \{
        return voiceVolume\_;
    \}
    Option<IntegerSpecifier>\& EffectVolume()
    \{
        return effectVolume\_;
    \}}\end{DoxyPre}



\begin{DoxyPre}{\ttfamily Nastaveni jednotne hlasitosti pomoci linku.
    void SetMasterVolume(int volume)
    \{
        musicVolume\_.Set(volume);
        voiceVolume\_.Link(musicVolume\_);
        effectVolume\_.Link(musicVolume\_);
    \}}\end{DoxyPre}



\begin{DoxyPre}{\ttfamily private:
    Option<IntegerSpecifier> musicVolume\_;
    Option<IntegerSpecifier> voiceVolume\_;
    Option<IntegerSpecifier> effectVolume\_;
\};}\end{DoxyPre}



\begin{DoxyPre}{\ttfamily Metoda pro hrani zvuku
void PlayMusic(int volume) \{\}}\end{DoxyPre}



\begin{DoxyPre}{\ttfamily int main()
\{
    Configuration configuration("game.ini", InputFilePolicy::IGNORE\_NONEXISTANT);}\end{DoxyPre}



\begin{DoxyPre}{\ttfamily     Game game(configuration);
     ... prace s vnejsim interfacem Game ...}\end{DoxyPre}



\begin{DoxyPre}{\ttfamily     Sound sound(configuration);
    PlayMusic(sound.MusicVolume().Get());
    sound.SetMasterVolume(75);
\}
}\end{DoxyPre}
 