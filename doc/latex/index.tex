Knihovna (C++) umožňuje přístup k hodnotám voleb v souboru (.ini).

\subsection*{Nástin použítí\+:}

Uživatel vytvoří „\+Configuration“ a předá ji jméno souboru. Uživatel vytvoří „\+Section“. Dalším krokem je specifikace voleb v sekci (instance „\+Option\+Proxy“,či „\+List\+Option\+Proxy“). Jejich hodnoty si pak uživatel uloží do vlastních proměnných/atributů. Při ukládání konfugurace jsou potom hodnoty v "Option\+Proxy“ uloženy také.

\subsection*{Mini příklad\+:}



 
\begin{DoxyPre}{\ttfamily 
Configuration config; 
config.open(inpuStream); // nyni se nactou data z daneho srteamu}\end{DoxyPre}



\begin{DoxyPre}{\ttfamily Section optionalSec = config.SpecifySection("optionalSection"); 
toto najde blok dat odpovidajici sekci nazvane "optionalSection"}\end{DoxyPre}



\begin{DoxyPre}{\ttfamily OptionProxy<StringSpecifier> optionFileName =  
            dummySection.SpecifyOption<StringSpecifier>("Save File", StringSpecifier(), "defaultFileName" );
nyni muze uzivatl prostrednictvim saveFileName cist a menit tuto volbu.}\end{DoxyPre}



\begin{DoxyPre}{\ttfamily ...}\end{DoxyPre}



\begin{DoxyPre}{\ttfamily optionFileName.Set("newFile.sav");}\end{DoxyPre}



\begin{DoxyPre}{\ttfamily ...}\end{DoxyPre}



\begin{DoxyPre}{\ttfamily cofing.save(outputStream); // timto se vsechny volby ulozi
}\end{DoxyPre}
 



\subsection*{Hlavní vlastnoti\+:}

Datové typy (podle kterých se mají textová data interpretovat) jsou určeny pomocí specialních tříd tzv.\char`\"{}\+Type\+Specfiers\char`\"{} (v příkladu \char`\"{}\+String\+Specifier\char`\"{}). Vlastní implementace těchto tříd umožuje uživateli definovát vlastní přechodové funkce mezi textem a požadovanými daty.

Třída „\+Configuration“ a „\+Option\+Proxy“ spolu komunikují po celou dobu. Po ukončení práce lze všechny aktuální hodnoty v specifikovaných \char`\"{}\+Option\+Proxy“ uložit pomocí \char`\"{}Configuration.\+save\char`\"{}. Konfliktům je zamezeno
tím, že od každé volby je vydána maximálně jedna instance \char`\"{}Option\+Proxy".

\subsection*{Příklad rendereru\+:}






\begin{DoxyPre}{\ttfamily 
class Upbp
\{
public:
    Upbp(Configuration& configuration)  
    \{
        Section sectionUpbp = configuration.SpecifySection("UPBP", Requirement::MANDATORY);
        beamDensity\_ = sectionUpbp.SpecifyOption("BeamDensity", FloatSpecifier(), 15.f);
    \}
    ...
\};
class Vcm
\{
public:
    Vcm(Configuration& configuration)
    \{
        Section sectionVcm = configuration.SpecifySection("VCM", Requirement::MANDATORY);
        radius\_ = sectionVcm.SpecifyOption("Radius", FloatSpecifier(), 15.f);
    \}
    ... 
\};}\end{DoxyPre}



\begin{DoxyPre}{\ttfamily enum AlgorithmType
\{
    UPBP,
    VCM
\}}\end{DoxyPre}



\begin{DoxyPre}{\ttfamily Configuration configuration("render.ini", InputFilePolicy::IGNORE\_NONEXISTANT);}\end{DoxyPre}



\begin{DoxyPre}{\ttfamily Section mainSection = configuration.SpecifySection("Main", Requirement::MANDATORY);
auto algorithmType = mainSection.SpecifyOption("Algorithm", EnumSpecifier<AlgorithmType>(\{ \{ "VCM", VCM \}, \{ "UPBP", UPBP \} \}), AlgorithmType::VCM);
switch (algorithmType.Get())
\{
    case VCM: renderer = new Vcm(configuration); break;
    case UPBP: renderer = new Upbp(configuration); break;
\}
}\end{DoxyPre}
 