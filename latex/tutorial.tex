\hypertarget{tutorial_Contents}{}\section{Contents}\label{tutorial_Contents}

\begin{DoxyItemize}
\item \hyperlink{tutorial_tutorial_introduction}{Introduction}
\item \hyperlink{tutorial_tutorial_tests}{Tests}
\begin{DoxyItemize}
\item \hyperlink{tutorial_tutorial_test_cases}{Test cases}
\item \hyperlink{tutorial_tutorial_asserts}{Asserts}
\end{DoxyItemize}
\item \hyperlink{tutorial_tutorial_output_handlers}{Output handlers}
\begin{DoxyItemize}
\item \hyperlink{tutorial_tutorial_available_output_handlers}{Available output handlers}
\item \hyperlink{tutorial_tutorial_screenshots}{Screenshots}
\end{DoxyItemize}
\item \hyperlink{tutorial_tutorial_test_suites}{Test suites}
\begin{DoxyItemize}
\item \hyperlink{tutorial_tutorial_running_test_suites}{Running test suites}
\item \hyperlink{tutorial_tutorial_embedded_test_suites}{Embedded test suites}
\item \hyperlink{tutorial_tutorial_test_fixtures}{Test fixtures}
\end{DoxyItemize}
\end{DoxyItemize}\hypertarget{tutorial_tutorial_introduction}{}\section{Introduction}\label{tutorial_tutorial_introduction}
Cpp\+Test is a portable and powerful, yet simple, unit testing framework for handling automated tests in C++. The focus lies on usability and extendability. Several output formats are supported and new ones are easily added.

This tutorial is intended to quickly get you started.

Cpp\+Test is released under the G\+NU \href{http://www.gnu.org/copyleft/lesser.html}{\tt Lesser General Public License}.\hypertarget{tutorial_tutorial_tests}{}\section{Tests}\label{tutorial_tutorial_tests}
\hypertarget{tutorial_tutorial_test_cases}{}\subsection{Test cases}\label{tutorial_tutorial_test_cases}
Each test must be created and run within a test suite (see \hyperlink{tutorial_tutorial_test_suites}{Test suites} below), which must be derived from \hyperlink{class_test_1_1_suite}{Test\+::\+Suite}. Tests must be registered with the \hyperlink{cpptest-suite_8h_abe8c3e0a2cf3893ebc1c265264ed9cb8}{T\+E\+S\+T\+\_\+\+A\+D\+D(func)} macro inorder to be executed. Note that a test function must return {\ttfamily void} and take no parameters. For example\+:


\begin{DoxyCode}
\textcolor{keyword}{class }ExampleTestSuite : \textcolor{keyword}{public} \hyperlink{class_test_1_1_suite}{Test::Suite}
\{
\textcolor{keyword}{public}:
    ExampleTestSuite()
    \{
        \hyperlink{cpptest-suite_8h_abe8c3e0a2cf3893ebc1c265264ed9cb8}{TEST\_ADD}(ExampleTestSuite::first\_test)
        \hyperlink{cpptest-suite_8h_abe8c3e0a2cf3893ebc1c265264ed9cb8}{TEST\_ADD}(ExampleTestSuite::second\_test)
    \}
    
private:
    \textcolor{keywordtype}{void} first\_test();
    \textcolor{keywordtype}{void} second\_test();
\};
\end{DoxyCode}
\hypertarget{tutorial_tutorial_asserts}{}\subsection{Asserts}\label{tutorial_tutorial_asserts}
Asserts are used to ensure that the tests are correct. If not, one or more asserts will be generated. There exist several types of asserts, see \hyperlink{asserts}{Available asserts} for a complete list.

For example, the functions declared within {\ttfamily Example\+Test\+Suite} may be declared as\+:


\begin{DoxyCode}
\textcolor{keywordtype}{void} ExampleTestSuite::first\_test()
\{
    \textcolor{comment}{// Will succeed since the expression evaluates to true}
    \textcolor{comment}{//}
    \hyperlink{cpptest-assert_8h_ac7dd7b06eb85d9ad841d80cbf217b1f6}{TEST\_ASSERT}(1 + 1 == 2)
    
    \textcolor{comment}{// Will fail since the expression evaluates to false}
    \textcolor{comment}{//}
    \hyperlink{cpptest-assert_8h_ac7dd7b06eb85d9ad841d80cbf217b1f6}{TEST\_ASSERT}(0 == 1);
\}

\textcolor{keywordtype}{void} ExampleTestSuite::second\_test()
\{
    \textcolor{comment}{// Will succeed since the expression evaluates to true}
    \textcolor{comment}{//}
    \hyperlink{cpptest-assert_8h_a9583b1709f4b9dfb3ff2849bfec5c885}{TEST\_ASSERT\_DELTA}(0.5, 0.7, 0.3);
    
    \textcolor{comment}{// Will fail since the expression evaluates to false}
    \textcolor{comment}{//}
    \hyperlink{cpptest-assert_8h_a9583b1709f4b9dfb3ff2849bfec5c885}{TEST\_ASSERT\_DELTA}(0.5, 0.7, 0.1);
\}
\end{DoxyCode}
\hypertarget{tutorial_tutorial_output_handlers}{}\section{Output handlers}\label{tutorial_tutorial_output_handlers}
\hypertarget{tutorial_tutorial_available_output_handlers}{}\subsection{Available output handlers}\label{tutorial_tutorial_available_output_handlers}
An output handler takes care of all assert messages and generates some output as result. There exist several output handlers that all fit different needs. Currently, the following output handlers exist\+:
\begin{DoxyItemize}
\item \hyperlink{class_test_1_1_text_output}{Test\+::\+Text\+Output}, which is a simple output handler that outputs its result to standard out in either terse or verbose mode.
\item \hyperlink{class_test_1_1_compiler_output}{Test\+::\+Compiler\+Output}, which emulates the output of a compiler. This makes it easy to integrate unit testing into your build environment.
\item \hyperlink{class_test_1_1_html_output}{Test\+::\+Html\+Output}, which outputs a fancy H\+T\+ML table.
\end{DoxyItemize}

New output handlers should be derived from \hyperlink{class_test_1_1_output}{Test\+::\+Output}.\hypertarget{tutorial_tutorial_screenshots}{}\subsection{Screenshots}\label{tutorial_tutorial_screenshots}
The result from the different output handlers is shown below\+:


\begin{DoxyItemize}
\item \href{screenshot-text-terse.png}{\tt Terse text output}
\item \href{screenshot-text-verbose.png}{\tt Verbose text output}
\item \href{screenshot-compiler.png}{\tt Compiler output}
\item \href{html-example.html}{\tt H\+T\+ML output}
\end{DoxyItemize}\hypertarget{tutorial_tutorial_test_suites}{}\section{Test suites}\label{tutorial_tutorial_test_suites}
\hypertarget{tutorial_tutorial_running_test_suites}{}\subsection{Running test suites}\label{tutorial_tutorial_running_test_suites}
The tests within a test suite are all executed when \hyperlink{class_test_1_1_suite_ad17746e218da79c537bc9d21e389f570}{Test\+::\+Suite\+::run()} is called. This method returns a boolean value that only returns true if all tests were successful. This value may be used to return a suiteable value from the program. For example, the following is required to execute our tests within {\ttfamily Example\+Test\+Suite}\+:


\begin{DoxyCode}
\hyperlink{class_test_1_1_text_output}{Test::TextOutput} output(\hyperlink{class_test_1_1_text_output_ae7b22c9458e6c566996bf4517c73feb1a85dd6e42f6261a23fd504201f5cc2792}{Test::TextOutput::Verbose});
ExampleTestSuite ets;
\textcolor{keywordflow}{return} ets.run(output) ? EXIT\_SUCCESS : EXIT\_FAILURE;
\end{DoxyCode}


Note that a single test normally continues after an assert. However, this behavior may be changed with an optional parameter to \hyperlink{class_test_1_1_suite_ad17746e218da79c537bc9d21e389f570}{Test\+::\+Suite\+::run()}. For example\+:


\begin{DoxyCode}
\textcolor{keyword}{class }SomeTestSuite: \textcolor{keyword}{public} \hyperlink{class_test_1_1_suite}{Test::Suite}
\{
\textcolor{keyword}{public}:
    SomeTestSuite() \{ \hyperlink{cpptest-suite_8h_abe8c3e0a2cf3893ebc1c265264ed9cb8}{TEST\_ADD}(SomeTestSuite::test) \}
\textcolor{keyword}{private}:
    \textcolor{keywordtype}{void} test()
    \{
        \hyperlink{cpptest-assert_8h_a947ab44cc42369eb7cfe33f8a1e38e4b}{TEST\_FAIL}(\textcolor{stringliteral}{"this will always fail"})
        \hyperlink{cpptest-assert_8h_a947ab44cc42369eb7cfe33f8a1e38e4b}{TEST\_FAIL}("this assert will never be executed")
    \}
\};

\textcolor{keywordtype}{bool} run\_tests()
\{
    SomeTestSuite sts;
    \hyperlink{class_test_1_1_text_output}{Test::TextOutput} output(\hyperlink{class_test_1_1_text_output_ae7b22c9458e6c566996bf4517c73feb1a85dd6e42f6261a23fd504201f5cc2792}{Test::TextOutput::Verbose});
    \textcolor{keywordflow}{return} sts.run(output, \textcolor{keyword}{false}); \textcolor{comment}{// Note the 'false' parameter}
\}
\end{DoxyCode}
\hypertarget{tutorial_tutorial_embedded_test_suites}{}\subsection{Embedded test suites}\label{tutorial_tutorial_embedded_test_suites}
\hyperlink{namespace_test}{Test} suites may also be added other test suites using \hyperlink{class_test_1_1_suite_a0237b63fc694ecb133d023cf2d6ab271}{Test\+::\+Suite\+::add()}. This way, you may develop logically separated test suites and run all tests at once. For example\+:


\begin{DoxyCode}
\textcolor{keyword}{class }TestSuite1: \textcolor{keyword}{public} \hyperlink{class_test_1_1_suite}{Test::Suite} \{ \}; \textcolor{comment}{// ... with many tests}
\textcolor{keyword}{class }TestSuite2: \textcolor{keyword}{public} \hyperlink{class_test_1_1_suite}{Test::Suite} \{ \}; \textcolor{comment}{// ... with many tests}
\textcolor{keyword}{class }TestSuite3: \textcolor{keyword}{public} \hyperlink{class_test_1_1_suite}{Test::Suite} \{ \}; \textcolor{comment}{// ... with many tests}

\textcolor{keywordtype}{bool} run\_tests()
\{
    \hyperlink{class_test_1_1_suite}{Test::Suite} ts;
    ts.\hyperlink{class_test_1_1_suite_a0237b63fc694ecb133d023cf2d6ab271}{add}(auto\_ptr<Test::Suite>(\textcolor{keyword}{new} TestSuite1));
    ts.\hyperlink{class_test_1_1_suite_a0237b63fc694ecb133d023cf2d6ab271}{add}(auto\_ptr<Test::Suite>(\textcolor{keyword}{new} TestSuite2));
    ts.\hyperlink{class_test_1_1_suite_a0237b63fc694ecb133d023cf2d6ab271}{add}(auto\_ptr<Test::Suite>(\textcolor{keyword}{new} TestSuite3));
    
    \hyperlink{class_test_1_1_text_output}{Test::TextOutput} output(\hyperlink{class_test_1_1_text_output_ae7b22c9458e6c566996bf4517c73feb1a85dd6e42f6261a23fd504201f5cc2792}{Test::TextOutput::Verbose});
    \textcolor{keywordflow}{return} ts.\hyperlink{class_test_1_1_suite_ad17746e218da79c537bc9d21e389f570}{run}(output);
\}
\end{DoxyCode}
\hypertarget{tutorial_tutorial_test_fixtures}{}\subsection{Test fixtures}\label{tutorial_tutorial_test_fixtures}
\hyperlink{namespace_test}{Test} cases in the same test suite often share the same requirements, they all require some known set of objects or resources. Instead of repeating all initialization code for each test it may be moved to the \hyperlink{class_test_1_1_suite_afb4c733e6c46a011818bb02f2e8d5bb8}{Test\+::\+Suite\+::setup()} and \hyperlink{class_test_1_1_suite_a37d3595625cff09b8e43bf6c414ff610}{Test\+::\+Suite\+::tear\+\_\+down()} functions. The setup function is called before each test function is called, and the tear down function is called after each test function has been called. For example\+:


\begin{DoxyCode}
\textcolor{keyword}{class }SomeTestSuite: \textcolor{keyword}{public} \hyperlink{class_test_1_1_suite}{Test::Suite}
\{
\textcolor{keyword}{public}:
    SomeTestSuite() 
    \{ 
        \hyperlink{cpptest-suite_8h_abe8c3e0a2cf3893ebc1c265264ed9cb8}{TEST\_ADD}(SomeTestSuite::test1) 
        \hyperlink{cpptest-suite_8h_abe8c3e0a2cf3893ebc1c265264ed9cb8}{TEST\_ADD}(SomeTestSuite::test2) 
    \}
protected:
    virtual \textcolor{keywordtype}{void} setup()     \{\} \textcolor{comment}{// setup resources... }
    \textcolor{keyword}{virtual} \textcolor{keywordtype}{void} \hyperlink{class_test_1_1_suite_a37d3595625cff09b8e43bf6c414ff610}{tear\_down}() \{\} \textcolor{comment}{// remove resources...}
\textcolor{keyword}{private}:
    \textcolor{keywordtype}{void} test1() \{\} \textcolor{comment}{// use common resources...}
    \textcolor{keywordtype}{void} test2() \{\} \textcolor{comment}{// use common resources...}
\};
\end{DoxyCode}
 